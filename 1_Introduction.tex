\section{Introduction}
\label{Sec:Introduction}

Position and speed detection system used in the automotive, security and surveillance sector typically use a Continuous Wave (CW) or Frequency Modulated Continuous Wave (FMCW) Radar. In these systems, the baseband processors analyse the Doppler information in the spectrum of the received signal to accurately estimate speed and position of the target. Discrete Fourier Transform is an important tool required to perform spectral analysis of the data that has a time varying component in its frequency. One of the most common methods of computing the DFT of a signal is using the Fast Fourier transform (FFT). FFT accelerators available on Digital Signal Processors, use a pipelined architecture to compute the FFT. These pipelined architectures are especially suitable for this purpose since they provide high throughput and low power, as well as low latency\cite{Mookherjee2015}. In addition to the pipelining, in systems where high throughput is a critical requirement, several samples of an input signal may be processed in parallel leading to the development of the 2-parallel Radix-2 \cite{Ayinala2012} and 8-parallel Radix-2 \cite{Mookherjee2015} algorithms. 

However for position and speed detection systems that have to work in real time, the frequency difference between a transmitted signal and its received reflection is used to detect the target range and velocity. The return signal, which has a different frequency, is called a ‘beat-signal’. Depending on the speed of the moving target, the bandwidth of the received beat-signal decreases by a few kiloHertz regardless of the transmitted bandwidth, the complexity of the signal processing can be reduced, compared with that of conventional spectrum analysis.and are concerned only with a small time-varying subset of the frequency spectrum around the beat frequency, the DFT needs to be calculated almost every sample. In such a scenario, the FFT computation scheme is not the best solution because it iig essentially a batch process and computing the $N$-point FFT requires the system to wait for $N$ time samples. 

The sliding window DFT (SDFT) is a popular alternative algorithm that was introduced by Springer \cite{Springer1988} and later improved by Jacobsen et al. \cite{Jacobsen2003}. The algorithm is efficient however, it is limited in that it is only marginally stable and requires storing $N$
previous samples. Grado et al. \cite{Grado2017} introduced the idea of a Sliding Windowed Infinite Fourier Transform (SWIFT) that has several advantages over the SDFT. The proposed algorithm is guaranteed stable and has an improved frequency domain sampling. In addition, unlike the SDFT that uses a rectangular window, the SWIFT uses an exponentially decaying window that assigns greater weights to recent samples. 

In recent years, frequency-modulated continuous-wave (FMCW) radars have been used in vehicle applications. In FMCW radars, linear ramps are generated and transmitted. The frequency difference between a transmitted signal and its received reflection is used to detect the target range and velocity. The return signal, which has a different frequency, is called a ‘beat-signal’. Because the bandwidth of the received beat-signal decreases by less than a dozen MHz regardless of the transmitted bandwidth, the complexity of the signal processing can be reduced, compared with that of conventional spectrum analysis.